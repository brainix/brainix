\documentclass{article}
\usepackage{code}
\title{How to run Brainix on Unix-like Operating Systems and Bochs}
\author{pqnelson}
\date{\today}
\begin{document}
 \maketitle

OK, you have downloaded your copy of brainix by first installing subversion (you need to install subversion!), then type in:

\verb|$ svn checkout http://brainix.googlecode.com/svn/trunk/ brainix|

It should start downloading. You wait until its downloading is complete. I assume that the working directory is \verb|/home/your_user_name_here/|. You type into the command line:

\verb|$ mkdir mnt|

Then go into the Brainix directory. Create several scripts:
\begin{code}{brainix/compile.sh}
#!/usr/bin/bash

make clean
make
\end{code}
The \verb|compile.sh| script.
\begin{code}{brainix/update\_bootdisk.sh}
#!/usr/bin/bash

sudo mount -o loop Bootdisk.img ../mnt/
sudo cp ./bin/Brainix ../mnt
sudo rm ../mnt/Brainix.gz
sudo gzip ../mnt/Brainix
sudo umount ../mnt/
\end{code}
This is the \verb|update_bootdisk| script that updates the boot disk with the Brainix binary image (the Brainix binary image is made after compilation and put in /brainix/bin). Now, to make a bochs \verb|bochsrc| file. I have included a \verb|dot-bochsrc| file in this documentation (its external to this file), I use it to run Brainix in bochs.

Now to put it all together:
\begin{code}{brainix/run.sh}
#!/usr/bin/bash

sh compile.sh
sudo sh update-bootdisk.sh
bochs -f dot-bochsrc
\end{code}
This is the script you invoke in order to compile and run the compiled image in bochs.

I have included sample scripts \textbf{THAT YOU NEED TO MOVE TO THE BRAINIX FOLDER}, i.e. type into the command line:

\verb|$ mv compile.sh ../|
\verb|$ mv update-bootdisk.sh ../|
\verb|$ mv dot-bochsrc ../|
\verb|$ mv run.sh ../|

\end{document}
